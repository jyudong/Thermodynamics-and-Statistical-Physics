\documentclass[a4paper,12pt]{article}

\usepackage{titlesec}
%\usepackage{geometry}
\usepackage{amsmath}   %\nonumber方程后不编号
\usepackage{xeCJK}
\setCJKmainfont{Kai}

\title{热力学统计物理第八次作业}
\date{2021.11}
\author{董建宇}

\begin{document}

\maketitle

\titleformat{\section}[hang]{\small}{\thesection}{0.8em}{}{}
\titleformat{\subsection}[hang]{\small}{\thesubsection}{0.8em}{}{}

\section{}
对于一维自由粒子有
\begin{equation}\nonumber
	p_x = \frac{2\pi\hbar}{L}n_x
\end{equation}
即
\begin{equation}\nonumber
	\,dn_x = \frac{L}{2\pi\hbar}\,dp_x = \frac{L}{h} \sqrt{\frac{m}{2\varepsilon}} \,d\varepsilon.
\end{equation}
由于动量大小相同时方向可以相反,则在$\varepsilon$到$\varepsilon + d\varepsilon$的能量范围内,量子态数为:
\begin{equation}\nonumber
	d\Omega = D(\varepsilon)\,d\varepsilon = 2\frac{L}{h} \,dp_x = \frac{2L}{h} \sqrt{\frac{m}{2\varepsilon}} \,d\varepsilon.
\end{equation}


\section{}
对于二维粒子有:
\begin{equation}\nonumber
	\,dn_x\,dn_y = \left(\frac{L}{2\pi\hbar}\right)^2\,dp_x\,dp_y = \left( \frac{L}{h} \right)^2 \,dp_x\,dp_y
\end{equation}
则在面积$L^2$内,在$\varepsilon$到$\varepsilon + d\varepsilon$的能量范围内,量子态数目为
\begin{equation}\nonumber
	\,d\Omega = D(\varepsilon) \,d\varepsilon = \frac{L^2}{\hbar^2} 2\pi p\,dp = \frac{2\pi L^2}{h^2}\,d\left( \frac{1}{2}p^2 \right) = \frac{2\pi L^2}{h^2}m\,d\varepsilon
\end{equation}


\section{}
在极端相对论情形下下有:
\begin{equation}\nonumber
	p = \frac{\varepsilon}{c},~\,dp = \frac{1}{c}\,d\varepsilon.
\end{equation}
在体积$V$的范围内,能量从$\varepsilon$到$\varepsilon + \,d\varepsilon$的能量范围内三维粒子的量子态数目为:
\begin{equation}\nonumber
	\,d\Omega = \frac{V}{h^3} 4\pi p^2\,dp = \frac{4\pi V}{h^3}\frac{\varepsilon^2}{c^3}\,d\varepsilon.
\end{equation}


\section{}
\subsection{n=0:}
当n=0时,量子态只能为$\vec{n} = (0,0,0)$,即简并度为$g(0) = 1$。
\subsection{n=1:}
当n=1时,量子态可能为$\vec{n} = (1,0,0)$、$\vec{n} = (-1,0,0)$、$\vec{n} = (0,1,0)$、$\vec{n} = (0,-1,0)$、$\vec{n} = (0,0,1)$、$\vec{n} = (0,0,-1)$,即简并度为$g(1) = 3 \times 2^1 = 6$。
\subsection{n=2:}
当n=2时,量子态可能为$\vec{n} = (1,1,0)$、$\vec{n} = (1,-1,0)$、$\vec{n} = (-1,1,0)$、$\vec{n} = (-1,-1,0)$、$\vec{n} = (1,0,1)$、$\vec{n} = (1,0,-1)$、$\vec{n} = (-1,0,1)$、$\vec{n} = (-1,0,-1)$、$\vec{n} = (0,1,1)$、$\vec{n} = (0,1,-1)$、$\vec{n} = (0,-1,1)$、$\vec{n} = (0,-1,-1)$,即简并度为$g(2) = 3 \times 2^2 = 12$。
\subsection{n=3:}
当n=3时,量子态可能为$\vec{n} = (1,1,1)$、$\vec{n} = (-1,1,1)$、$\vec{n} = (1,-1,1)$、$\vec{n} = (1,1,-1)$、$\vec{n} = (-1,-1,1)$、$\vec{n} = (-1,1,-1)$、$\vec{n} = (1,-1,-1)$、$\vec{n} = (-1,-1,-1)$,即简并度为$g(3) = 1 \times 2^3 = 8$。
\subsection{n=4:}
当n=4时,量子态可能为$\vec{n} = (2,0,0)$、$\vec{n} = (-2,0,0)$、$\vec{n} = (0,2,0)$、$\vec{n} = (0,-2,0)$、$\vec{n} = (0,0,2)$、$\vec{n} = (0,0,-2)$,即简并度为$g(4) = 3 \times 2^1 = 6$。


\section{}
\subsection{}
设Y为向右移动的步数,则X服从二项分布参数为(N,p),则原子在$t = N \tau$时刻的位置为:
\begin{equation}\nonumber
	x = aY-a(N-Y) = 2aY - aN
\end{equation}
则原子的平均位置为:
\begin{equation}\nonumber
	\bar{x} = 2a\bar{Y} - aN
\end{equation}
由二项分布的性质可知:$\bar{Y} = Np$,则原子在$t = N\tau$时刻的平均位置为:
\begin{equation}\nonumber
	\bar{x} = aN(2p - 1).
\end{equation}
\subsection{}
t时刻原子平均位置的方均偏差为:
\begin{equation}\nonumber
	\mathbf{Var}(x) = \overline{(x - \bar{x})^2} = 4a^2 \mathbf{Var}(Y)
\end{equation}
由二项分布性质可知:$\mathbf{Var}(Y) = Np(1-p)$,则t时刻原子平均位置的方均偏差为:
\begin{equation}\nonumber
	\overline{(x - \bar{x})^2} = 4a^2Np(1-p).
\end{equation}



\end{document}
