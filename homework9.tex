\documentclass[a4paper,12pt]{article}

\usepackage{titlesec}
%\usepackage{geometry}
\usepackage{amsmath}   %\nonumber方程后不编号
\usepackage{amsthm,amsmath,amssymb}
\usepackage{mathrsfs}
\usepackage{xeCJK}
\usepackage{graphicx}
%\usepackage{braket}
\setCJKmainfont{Kai}

\title{热力学统计物理第九次作业}
\date{2021.11}
\author{董建宇}


\begin{document}
\maketitle 

\titleformat{\section}[hang]{\large}{\thesection}{0.8em}{}{}
\titleformat{\subsection}[hang]{\small}{\thesubsection}{0.8em}{}{}

\section{}
由于两种粒子可以看做近独立粒子且可以分辨,则粒子数为$N$的粒子1的微观状态数$\Omega_1$和粒子数为$N'$的粒子2的微观状态数$\Omega_2$分别为:
\begin{equation}\nonumber
	\Omega_1 = \frac{N!}{\prod\limits_{l}a_l!}\prod_{l}\omega_l\,^{a_l}, ~ \Omega_2 = \frac{N'!}{\prod\limits_l a_l'!}\prod_l \omega_l'\,^{a_l'}.
\end{equation}
则系统的总状态数为:
\begin{equation}\nonumber
	\Omega = \Omega_1 \times \Omega_2 = \left( \frac{N!}{\prod\limits_{l}a_l!}\prod_{l}\omega_l\,^{a_l} \right) \times \left( \frac{N'!}{\prod\limits_l a_l'!}\prod_l \omega_l'\,^{a_l'} \right).
\end{equation}
两侧取对数得:
\begin{equation}\nonumber
	\ln \Omega = \ln N! + \ln N'! - \sum_l \ln a_l! + \sum_l a_l \ln \omega_l - \sum_l \ln a_l'! + \sum_l a_l' \omega_l'.
\end{equation}
当所有的$a_l,a_l'$都很大时有:
\begin{equation}\nonumber
\begin{aligned}
	&\ln a_l! =a_l \left( \ln a_l - 1 \right), & &\ln a_l' = a_l' \left( \ln a_l' - 1 \right), \\
	&\ln N! = N \left( \ln N - 1 \right), & &\ln N'! = N' \left( \ln N' - 1 \right)
\end{aligned}
\end{equation}
在平衡条件下有:
\begin{equation}\nonumber
\begin{aligned}
	&\delta N = \sum_l \delta a_l = 0, \delta N' = \sum_l \delta a_l' = 0, \\
	&\delta E = \sum_l \varepsilon_l a_l + \sum_l \varepsilon_l' a_l' =0, \\
	&\delta \ln \Omega = -\sum_l \ln\left( \frac{a_l}{\omega_l} \right) \delta a_l + \sum_l \ln\left( \frac{a_l'}{\omega_l'} \right) \delta a_l'.
\end{aligned}
\end{equation}
因而对于任意的$\alpha$,$\alpha'$和$\beta$都有:
\begin{equation}\nonumber
\begin{aligned}
	&\delta \ln \Omega - \alpha \delta N - \alpha' \delta N' - \beta \delta E \\
	=& -\sum_l \left[ \ln \left( \frac{a_l}{\omega_l} \right) + \alpha + \beta \varepsilon_l \right] \delta a_l - \sum_l \left[ \ln \left( \frac{a_l'}{\omega_l'} \right) + \alpha' + \beta \varepsilon_l' \right] \delta a_l' = 0
\end{aligned}
\end{equation}
由于$\delta a_l$和$\delta a_l'$可以任意取值,则有:
\begin{equation}\nonumber
	\ln \left( \frac{a_l}{\omega_l} \right) + \alpha + \beta \varepsilon_l = 0,~ \ln \left( \frac{a_l'}{\omega_l'} \right) + \alpha' + \beta \varepsilon_l' = 0.
\end{equation}
即
\begin{equation}\nonumber
	a_l = \omega_l e^{-\alpha - \beta \varepsilon_l}, ~ a_l' = \omega_l' e^{-\alpha' - \beta \varepsilon_l'}.
\end{equation}
此时系统的状态数为:
\begin{equation}\nonumber
\begin{aligned}
	\Omega = \left( \frac{N!}{\prod\limits_l \omega_le^{-\alpha-\beta\varepsilon_l}!}\prod\limits_l \omega_l^{\omega_l^{-\alpha-\beta\varepsilon_l}} \right) \times \left( \frac{N'!}{\prod\limits_l \omega_l'e^{-\alpha'-\beta\varepsilon_l'}!}\prod\limits_l \omega_l'^{\omega_l'^{-\alpha'-\beta\varepsilon_l'}} \right)
\end{aligned}
\end{equation}

如果将一种粒子看作一个子系统,有:
\begin{equation}\nonumber
	a_l = \omega_l e^{-\alpha - \beta \varepsilon_l}, ~ a_l' = \omega_l' e^{-\alpha' - \beta' \varepsilon_l'}
\end{equation}
则当两系统处于热平衡时有$\beta = \beta'$,即具有共同的$\beta$。


\section{}
\subsection{粒子是玻色子:}
粒子1的微观状态数$\Omega_1$和粒子2的微观状态数$\Omega_2$分别为:
\begin{equation}\nonumber
	\Omega_1 = \prod_l \frac{(\omega_l + a_l - 1)!}{a_l!(\omega_l - 1)!}, ~ \Omega_2 = \prod_l \frac{(\omega_l' + a_l' - 1)!}{a_l'!(\omega_l' - 1)!}.
\end{equation}
则系统的微观状态数为:
\begin{equation}\nonumber
	\Omega = \Omega_1 \times \Omega_2 = \left( \prod_l \frac{(\omega_l + a_l - 1)!}{a_l!(\omega_l - 1)!} \right) \times \left( \prod_l \frac{(\omega_l' + a_l' - 1)!}{a_l'!(\omega_l' - 1)!} \right). 
\end{equation}
两侧取对数可得:
\begin{equation}\nonumber
\begin{aligned}
	\ln \Omega =& \sum_l \left( \ln(\omega_l + a_l -1)! - \ln a_l! - \ln (\omega_l - 1)! \right)\\ 
	&+ \sum_l \left( \ln(\omega_l' + a_l' - 1)! - \ln a_l'! -\ln (\omega_l' - 1)! \right).
\end{aligned}
\end{equation}
当$a_l>>1,\omega_l>>1;a_l'>>1,\omega_l'>>1$时,有:
\begin{equation}\nonumber
\begin{aligned}
	\ln \Omega =& \sum_l \left( (\omega_l + a_l)\ln(\omega_l + a_l) - a_l \ln a_l - \omega_l \ln \omega_l \right) \\ 
	&+ \sum_l \left( (\omega_l' + a_l')\ln(\omega_l' + a_l') - a_l' \ln a_l' - \omega_l' \ln \omega_l' \right).
\end{aligned}
\end{equation}
则在平衡条件下有:
\begin{equation}\nonumber
\begin{aligned}
	&\delta N = \sum_l \delta a_l = 0, ~ \delta N' = \sum_l \delta a_l', \\
	&\delta E = \sum_l \varepsilon_l a_l + \sum_l \varepsilon_l' a_l' = 0, \\
	&\delta \Omega = \sum_l \ln\left( \frac{\omega_l}{a_l} + 1 \right) \delta a_l + \sum_l \ln \left( \frac{\omega_l'}{a_l'} + 1 \right) \delta a_l'.
\end{aligned}
\end{equation}
则对于任意的$\alpha,\alpha'$和$\beta$都有:
\begin{equation}\nonumber
\begin{aligned}
	&-\delta \Omega + \alpha \delta N + \alpha' \delta N' + \beta \delta E \\
	=& \sum_l \left[ -\ln \left( \frac{\omega_l}{a_l} + 1 \right) + \alpha + \beta \varepsilon_l \right] \delta a_l + \sum_l \left[ -\ln \left( \frac{\omega_l'}{a_l'} + 1 \right) + \alpha' + \beta \varepsilon_l' \right] \delta a_l'
\end{aligned}
\end{equation}
由于$\delta a_l$和$\delta a_l'$可以取任意值,则有:
\begin{equation}\nonumber
	-\ln \left( \frac{\omega_l}{a_l} + 1 \right) + \alpha + \beta\varepsilon_l = 0, ~ -\ln \left( \frac{\omega_l'}{a_l'} + 1 \right) + \alpha' + \beta \varepsilon_l' = 0.
\end{equation}
即
\begin{equation}\nonumber
	a_l = \frac{\omega_l}{e^{\alpha + \beta\varepsilon_l} - 1}, ~ a_l' = \frac{\omega_l'}{e^{\alpha' + \beta \varepsilon_l'} - 1}.
\end{equation}
此时系统微观状态数为:
\begin{equation}\nonumber
	\Omega = \left( \prod_l \frac{(\omega_l + \frac{\omega_l}{e^{\alpha+\beta\varepsilon_l} - 1} - 1)!}{(\frac{\omega_l}{e^{\alpha + \beta\varepsilon_l} - 1})! (\omega_l - 1)!} \right) \times \left( \prod_l \frac{(\omega_l' + \frac{\omega_l'}{e^{\alpha'+\beta\varepsilon_l'} - 1} - 1)!}{(\frac{\omega_l'}{e^{\alpha' + \beta\varepsilon_l'} - 1})! (\omega_l' - 1)!} \right)
\end{equation}

\subsection{粒子是费米子:}
粒子1的微观状态数$\Omega_1$和粒子2的微观状态数$\Omega_2$分别为:
\begin{equation}\nonumber
	\Omega_1 = \prod_l \frac{\omega_l!}{a_l!(\omega_l - a_l)!}, ~ \Omega_2 = \prod_l \frac{\omega_l'!}{a_l'!(\omega_l' - a_l')!}.
\end{equation}
则系统的微观状态数为:
\begin{equation}\nonumber
	\Omega = \Omega_1 \times \Omega_2 = \left( \prod_l \frac{\omega_l!}{a_l!(\omega_l - a_l)!} \right) \times \left( \prod_l \frac{\omega_l'!}{a_l'!(\omega_l' - a_l)!} \right). 
\end{equation}
两侧取对数可得:
\begin{equation}\nonumber
\begin{aligned}
	\ln \Omega =& \sum_l \left( \ln\omega_l! - \ln a_l! - \ln (\omega_l - a_l)! \right)\\ 
	&+ \sum_l \left( \ln\omega_l'! - \ln a_l'! -\ln (\omega_l' - a_l)! \right).
\end{aligned}
\end{equation}
当$a_l>>1,\omega_l>>1,\omega_l-a_l?>>1;a_l'>>1,\omega_l'>>1,\omega_l'-a_l>>1$时,有:
\begin{equation}\nonumber
\begin{aligned}
	\ln \Omega =& \sum_l \left( \omega_l\ln\omega_l - a_l \ln a_l - (\omega_l - a_l)\ln (\omega_l - a_l) \right) \\ 
	&+ \sum_l \left( \omega_l'\ln\omega_l' - a_l' \ln a_l' - (\omega_l' - a_l')\ln (\omega_l' - a_l) \right).
\end{aligned}
\end{equation}
则在平衡条件下有:
\begin{equation}\nonumber
\begin{aligned}
	&\delta N = \sum_l \delta a_l = 0, ~ \delta N' = \sum_l \delta a_l', \\
	&\delta E = \sum_l \varepsilon_l a_l + \sum_l \varepsilon_l' a_l' = 0, \\
	&\delta \Omega = \sum_l \ln\left( \frac{\omega_l}{a_l} - 1 \right) \delta a_l + \sum_l \ln \left( \frac{\omega_l'}{a_l'} - 1 \right) \delta a_l'.
\end{aligned}
\end{equation}
则对于任意的$\alpha,\alpha'$和$\beta$都有:
\begin{equation}\nonumber
\begin{aligned}
	&-\delta \Omega + \alpha \delta N + \alpha' \delta N' + \beta \delta E \\
	=& \sum_l \left[ -\ln \left( \frac{\omega_l}{a_l} - 1 \right) + \alpha + \beta \varepsilon_l \right] \delta a_l + \sum_l \left[ -\ln \left( \frac{\omega_l'}{a_l'} - 1 \right) + \alpha' + \beta \varepsilon_l' \right] \delta a_l'
\end{aligned}
\end{equation}
由于$\delta a_l$和$\delta a_l'$可以取任意值,则有:
\begin{equation}\nonumber
	-\ln \left( \frac{\omega_l}{a_l} - 1 \right) + \alpha + \beta\varepsilon_l = 0, ~ -\ln \left( \frac{\omega_l'}{a_l'} - 1 \right) + \alpha' + \beta \varepsilon_l' = 0.
\end{equation}
即
\begin{equation}\nonumber
	a_l = \frac{\omega_l}{e^{\alpha + \beta\varepsilon_l} + 1}, ~ a_l' = \frac{\omega_l'}{e^{\alpha' + \beta \varepsilon_l'} + 1}.
\end{equation}
此时系统的微观状态数为:
\begin{equation}\nonumber
	\Omega = \left( \prod_l \frac{\omega_l!}{(\frac{\omega_l}{e^{\alpha+\beta\varepsilon_l}+1})!(\omega_l - \frac{\omega_l}{e^{\alpha+\beta\varepsilon_l}+1})!} \right) \times \left( \prod_l \frac{\omega_l'!}{(\frac{\omega_l'}{e^{\alpha'+\beta\varepsilon_l'}+1})!(\omega_l' - \frac{\omega_l'}{e^{\alpha'+\beta\varepsilon_l'}+1})!} \right)
\end{equation}


\section{}
不妨设粒子1为玻色子,粒子2为费米子,则两种粒子的微观状态数分别为:
\begin{equation}\nonumber
	\Omega_1 = \prod_l \frac{(\omega_l + a_l - 1)!}{a_l!(\omega_l - 1)!}, ~ \Omega_2 = \prod_l \frac{\omega_l'!}{a_l!(\omega_l' - a_l')!}.
\end{equation}
则系统的微观状态数为:
\begin{equation}\nonumber
	\Omega = \Omega_1 \times \Omega_2 = \left( \prod_l \frac{(\omega_l + a_l - 1)!}{a_l!(\omega_l - 1)!} \right) \times \left( \prod_l \frac{\omega_l'!}{a_l!(\omega_l' - a_l')!} \right)
\end{equation}
两侧取对数可得:
\begin{equation}\nonumber
\begin{aligned}
	\ln \Omega =& \sum_l \ln(\omega_l + a_l - 1)! - \ln a_l! - \ln (\omega - 1)! \\ 
	&+ \sum_l \ln \omega_l'! - \ln a_l'! - \ln (\omega_l' - a_l')!.
\end{aligned}
\end{equation}
当$a_l>>1,\omega_l>>1;a_l'>>1,\omega_l'>>1,\omega_l'-a_l'>>1$时,有:
\begin{equation}\nonumber
\begin{aligned}
	\ln \Omega =& \sum_l \left( (\omega_l + a_l)\ln(\omega_l + a_l) - a_l\ln a_l - \omega_l \ln \omega_l \right) \\ 
	&+ \sum_l \left( \omega_l' \ln\omega_l' - a_l' \ln a_l' - (\omega_l' - a_l')\ln(\omega_l' - a_l') \right).
\end{aligned}
\end{equation}
在平衡条件下有:
\begin{equation}\nonumber
\begin{aligned}
	&\delta N = \sum_l \delta a_l = 0, ~ \delta N' = \sum_l \delta a_l' = 0, \\
	&\delta E = \sum_l \varepsilon_l \delta a_l + \sum_l \varepsilon_l' \delta a_l' = 0, \\
	&\delta \ln\Omega = \sum_l \ln\left( \frac{\omega_l}{a_l} + 1 \right) \delta a_l + \sum_l \ln\left( \frac{\omega_l'}{a_l'} - 1 \right) \delta a_l'.
\end{aligned}
\end{equation}
则对于任意的$\alpha, \alpha', \beta$都有:
\begin{equation}\nonumber
\begin{aligned}
	&-\delta \Omega + \alpha \delta N + \alpha' \delta N' + \beta \delta E \\
	=& \sum_l \left[ -\ln\left( \frac{\omega_l}{a_l} + 1 \right) + \alpha + \beta\varepsilon_l \right] \delta a_l + \sum_l \left[ -\ln\left( \frac{\omega_l'}{a_l'} - 1 \right) + \alpha' + \beta\varepsilon_l' \right] \delta a_l'
\end{aligned}
\end{equation}
由于$\delta a_l$和$\delta a_l'$可以取任意值,则有:
\begin{equation}\nonumber
	-\ln\left( \frac{\omega_l}{a_l} + 1 \right) + \alpha + \beta\varepsilon_l = 0, ~ -\ln\left( \frac{\omega_l'}{a_l'} - 1 \right) + \alpha' + \beta\varepsilon_l' = 0.
\end{equation}
即
\begin{equation}\nonumber
	a_l = \frac{\omega_l}{e^{\alpha + \beta\varepsilon_l} - 1}, ~ a_l' = \frac{\omega_l'}{e^{\alpha' + \beta\varepsilon_l'} + 1}.
\end{equation}
此时系统的微观状态数为:
\begin{equation}\nonumber
	\Omega = \left( \prod_l \frac{(\omega_l+\frac{\omega_l}{e^{\alpha+\beta\varepsilon_l}-1}-1)!}{(\frac{\omega_l}{e^{\alpha+\beta\varepsilon_l}-1})!(\omega_l-1)!} \right) \times \left( \prod_l \frac{(\omega_l'+\frac{\omega_l'}{e^{\alpha'+\beta\varepsilon_l'}-1}-1)!}{(\frac{\omega_l'}{e^{\alpha'+\beta\varepsilon_l'}-1})!(\omega_l'-1)!} \right)
\end{equation}



\section{}
当两能级的简并度均为1时,假设粒子可分辨,系统的熵为:
\begin{equation}\nonumber
	\Omega = \frac{N!}{N_1!N_2!}
\end{equation}
则系统的熵为:
\begin{equation}\nonumber
	S = k \ln\Omega = k \ln\left( \frac{N!}{N_1!N_2!} \right).
\end{equation}
当$N>>1,N_1>>1,N_2>>1$时,近似可得:
\begin{equation}\nonumber
	S = k\left(N\ln N - N_1\ln N_1 -N_2\ln N_2 \right).
\end{equation}


\section{}
粒子总数等于等于自旋向上的粒子数加自旋向下粒子数:
\begin{equation}\nonumber
	N = N_{\uparrow} + N_{\downarrow}.
\end{equation}
由提示可知:
\begin{equation}\nonumber
	M_z = \frac{1}{2}(N_{\uparrow} - N_{\downarrow}).
\end{equation}
所以可以解得:
\begin{equation}\nonumber
	N_{\uparrow} = \frac{N}{2} + M_z, ~ N_{\downarrow} = \frac{N}{2} - M_z.
\end{equation}
当简并度均为1时,体系状态数为:
\begin{equation}\nonumber
	\Omega = \binom{N}{N_{\uparrow}} = \frac{N!}{\left( \frac{N}{2} + M_z \right)! \left( \frac{N}{2} - M_z \right)!}.
\end{equation}
要使体系状态数最大,则要$\left( \frac{N}{2} + M_z \right)! \left( \frac{N}{2} - M_z \right)!$最小。绘图可知
\begin{center}
	\includegraphics[scale = 0.2]{hw9p5.jpeg} 
\end{center}
对于任意的$N$当$M_z = 0$时$\left( \frac{N}{2} + M_z \right)! \left( \frac{N}{2} - M_z \right)!$最小。即使体系状态数最大的$M_z$的值为0。\\
当$\frac{N}{2} + M_z$和$\frac{N}{2} - M_z$远大于1时有
\begin{equation}\nonumber
\begin{aligned}
	&\ln\left( \frac{N}{2} + M_z \right)! \approx \left( \frac{N}{2} + M_z \right) \left( \ln\left(\frac{N}{2} + M_z\right) - 1 \right) \\
	&\ln\left( \frac{N}{2} - M_z \right)! \approx \left( \frac{N}{2} - M_z \right) \left( \ln\left(\frac{N}{2} - M_z\right) - 1 \right)
\end{aligned}
\end{equation}
则有:
\begin{equation}\nonumber
\begin{aligned}
	\ln \Omega &= \ln N! - \ln\left( \frac{N}{2} + M_z \right)! - \ln\left( \frac{N}{2} - M_z \right)! \\
	&\approx N\ln N - \left( \frac{N}{2} + M_z \right) \ln\left(\frac{N}{2} + M_z\right) - \left( \frac{N}{2} - M_z \right) \ln\left(\frac{N}{2} - M_z\right)
\end{aligned}
\end{equation}
对$M_z$求导可得:
\begin{equation}\nonumber
	\frac{d \ln\Omega}{d M_z} = \ln\left( \frac{N - 2M_z}{N + 2M_z} \right) = 0.
\end{equation}
得$M_z = 0$,此时$\ln \Omega$取极值,此时有
\begin{equation}\nonumber
	\frac{d^2 \Omega}{d M_z^2} = -\frac{4}{N} < 0.
\end{equation}
即$M_z = 0$为$\ln\Omega$的极大值,即体系最状态数最大时$M_z = 0$,此时系统的微观状态数为:
\begin{equation}\nonumber
	\Omega_{max} = \frac{N!}{\left(\frac{N}{2}\right)!\left(\frac{N}{2}\right)!}
\end{equation}




\end{document}

注意到:
\begin{equation}\nonumber
\begin{aligned}
	&\left( \frac{N}{2} + M_z \right)! \left( \frac{N}{2} - M_z \right)! \\ 
	=& \left[ \prod_{l=0}^{\frac{N}{2} - M_z} \left( \frac{N}{2} + M_z - l \right) \left( \frac{N}{2} - M_z - l \right) \right]  \left[ \prod_{i=1}^{2M_z} \left( \frac{N}{2} - M_z + i \right) \right] \\
	=& \left[ \prod_{l=0}^{\frac{N}{2} - M_z} \left( \frac{N}{2} + M_z - l \right) \left( \frac{N}{2} - M_z - l \right) \right] \left[ \prod_{j=0}^{} \right]
\end{aligned}
\end{equation}

