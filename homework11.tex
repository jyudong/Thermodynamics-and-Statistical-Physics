\documentclass[a4paper,12pt]{article}

\usepackage{titlesec}
%\usepackage{geometry}
\usepackage{amsmath}   %\nonumber方程后不编号
\usepackage{amsthm,amsmath,amssymb}
\usepackage{mathrsfs}
\usepackage{xeCJK}
\usepackage{graphicx}
%\usepackage{braket}
\setCJKmainfont{Kai}

\title{热力学统计物理第十一次作业}
\date{2021.12}
\author{董建宇}


\begin{document}
\maketitle 

\titleformat{\section}[hang]{\large}{\thesection}{0.8em}{}{}
\titleformat{\subsection}[hang]{\small}{\thesubsection}{0.8em}{}{}

\section{}
\subsection{}
对于处于热力学平衡态的经典理想气体中的一个分子,速率分布函数为:
\begin{equation}\nonumber
	f_{\vec{V}}(\vec{v}) = 4\pi v^2 \left( \frac{m}{2\pi k_B T} \right)^{3/2} e^{-\frac{m v^2}{2k_BT}}.
\end{equation}
由$\varepsilon_1 = \frac{1}{2}mv^2,~d\varepsilon = mvdv$可知能量处于$\varepsilon_1$与$\varepsilon_1+d\varepsilon_1$之间的概率为:
\begin{equation}\nonumber
	\varphi_1(\varepsilon_1) d\varepsilon_1 = f_{\vec{V}}(\vec{v}) dv = \frac{2}{\sqrt{\pi}}\frac{\sqrt{\varepsilon_1}}{(k_BT)^{3/2}} e^{-\frac{\varepsilon_1}{k_BT}} d\varepsilon_1.
\end{equation}
同理可知,对于另一个处于热力学平衡的经典理想气体分子,能量处于$\varepsilon_2$与$\varepsilon_2+d\varepsilon_2$之间的概率为:
\begin{equation}\nonumber
	\varphi_2(\varepsilon_2) d\varepsilon_2 = \frac{2}{\sqrt{\pi}}\frac{\sqrt{\varepsilon_1}}{(k_BT)^{3/2}} e^{-\frac{\varepsilon_2}{k_BT}} d\varepsilon_2.
\end{equation}
则两个分子的总能量$\varepsilon = \varepsilon_1 + \varepsilon_2$的概率密度函数为:
\begin{equation}\nonumber
\begin{aligned}
	\psi(\varepsilon) &= \int_{0}^{\varepsilon} \varphi_1(\varepsilon-\varepsilon_2) \varphi_2(\varepsilon_2) \,d\varepsilon_2 \\
	&= \frac{4}{\pi}\frac{\varepsilon}{(k_BT)^3} e^{-\frac{\varepsilon}{k_BT}} \int_0^{\varepsilon} \sqrt{\frac{\varepsilon_2}{\varepsilon}}\sqrt{1-\frac{\varepsilon_2}{\varepsilon}} \,d\varepsilon_2 \\
	&= \frac{1}{\pi}\frac{\varepsilon^2}{(k_BT)^3} e^{-\frac{\varepsilon}{k_BT}} \int_0^{\pi/2} (1-\cos(4\theta)) \,d\theta \\
	&= \frac{1}{2} \frac{\varepsilon^2}{(k_BT)^3} e^{-\frac{\varepsilon}{k_BT}}.
\end{aligned}
\end{equation}
其中$\varepsilon_2 = \varepsilon \sin^2\theta$。也就是说两个分子总能量处在$\varepsilon$与$\varepsilon+d\varepsilon$之间的概率为:
\begin{equation}\nonumber
	\psi(\varepsilon)\,d\varepsilon = \frac{1}{2} \frac{\varepsilon^2}{(k_BT)^3} e^{-\frac{\varepsilon}{k_BT}}\,d\varepsilon.
\end{equation}
\subsection{}
两个分子总能量的平均值为:
\begin{equation}\nonumber
	\bar{\varepsilon} = \int_{0}^{+\infty} \frac{\varepsilon}{2} \frac{\varepsilon^2}{(k_BT)^3} e^{-\frac{\varepsilon}{k_BT}} \,d\varepsilon = \frac{1}{2(k_BT)^3} \times 6(k_BT)^4 = 3k_BT.
\end{equation}
即两个分子的总能量的平均值为$\bar{\varepsilon} = 3k_BT$。


\section{}
设$dE$为在$dt$时间内,碰撞到$dA$面积上,速度在$dv_x dv_y dv_z$范围内传给器壁的能量,则有:
\begin{equation*}
	dE dt dS = \alpha n \left( \frac{m}{2\pi k_BT} \right)^{3/2} e^{-\frac{m}{2k_BT}(v_x^2 + v_z^2 + v_z^2)} \frac{1}{2}m(v_x^2 + v_y^2 + v_z^2) v_x dv_x dv_y dv_z dt dS
\end{equation*}
利用求坐标变换,则有:
\begin{equation*}
	dE = \frac{\alpha nm}{2} \left( \frac{m}{2\pi k_BT} \right)^{3/2} v^5 e^{-\frac{m}{2k_BT} v^2} \sin^2\theta \cos\gamma \,dv \,d\theta \,d\gamma.
\end{equation*}
积分可知,传递能量为:
\begin{equation*}
\begin{aligned}
	E &= \frac{\alpha nm}{2} \left( \frac{m}{2\pi k_BT} \right)^{3/2} \int_0^{+\infty} \,dv \int_{0}^{\pi} \,d\theta \int_{-\pi/2}^{\pi/2} \,d\gamma v^5 e^{-\frac{m}{2k_BT} v^2} \sin^2\theta \cos\gamma \\
	&= \frac{\pi\alpha nm}{2} \left( \frac{m}{2\pi k_BT} \right)^{3/2} \left( \frac{2k_BT}{m} \right)^3 \\
	&= \sqrt{\frac{2}{\pi m}}\alpha n\left( k_BT \right)^{3/2}
\end{aligned}
\end{equation*}


\section{}
在温度T的热力学平衡状态下系统的配分函数为:
\begin{equation}\nonumber
	Z = e^{-\beta \varepsilon_1}+e^{-\beta \varepsilon_2}.
\end{equation}
则系统内能为:
\begin{equation}\nonumber
	U = -N\frac{\partial}{\partial \beta}\ln Z = N\frac{\varepsilon_1e^{-\beta\varepsilon_1}+\varepsilon_2e^{-\beta\varepsilon_2}}{e^{-\beta\varepsilon_1}+e^{-\beta\varepsilon_2}}.
\end{equation}
系统的熵为:
\begin{equation}\nonumber
\begin{aligned}
	S &= Nk\left( \ln Z-\beta\frac{\partial}{\partial\beta} \ln Z \right) \\
	&= Nk \left( \ln\left( e^{-\beta\varepsilon_1}+e^{-\beta\varepsilon_2} \right) + \beta \frac{\varepsilon_1e^{-\beta\varepsilon_1}+\varepsilon_2e^{-\beta\varepsilon_2}}{e^{-\beta\varepsilon_1}+e^{-\beta\varepsilon_2}} \right)
\end{aligned}
\end{equation}
在高温极限下,有$\beta \to 0$。则系统内能为:
\begin{equation}\nonumber
	U = N\frac{\varepsilon_1+\varepsilon_2}{2}.
\end{equation}
解释:粒子处在$\varepsilon_1$和$\varepsilon_2$能级上概率几乎相等,则有一半的粒子处于$\varepsilon_1$能级,有一半的粒子处于$\varepsilon_2$能级。 \\
系统的熵为:
\begin{equation}\nonumber
	S = Nk\ln2.
\end{equation}
解释:粒子处在$\varepsilon_1$和$\varepsilon_2$能级上概率几乎相等,由于每个粒子有两种可能的状态,系统的微观状态数为$2^N$。则熵为上式。 \\
在低温极限下,有$\beta \to \infty$。则系统内能为:
\begin{equation}\nonumber
	U = N \min\{\varepsilon_1,\varepsilon_2\}.
\end{equation}
其中$\min\{ \varepsilon_1,\varepsilon_2 \}$为取两者最小值。解释:低温极限下全部粒子处于低能级,则系统内能为粒子数乘以低能级能量。 \\
系统的熵为:
\begin{equation}\nonumber
	S = 0
\end{equation}
解释:低温极限下全部粒子处于低能级,则系统的微观状态数为1,因而系统的熵为0。 


\section{}
由题意可知,能量为$0$的能级间并度为1,能量为$\varepsilon$的能级间并度为2。则系统的配分函数为:
\begin{equation}\nonumber
	Z = 1+2e^{-\beta\varepsilon}.
\end{equation}
\subsection{}
则系统的熵为:
\begin{equation}\nonumber
\begin{aligned}
	S &= Nk\left( \ln Z - \beta\frac{\partial}{\partial \beta}\ln Z \right) \\ 
	&= Nk\left( \ln\left( 1+2e^{-\beta\varepsilon} \right) + \frac{2\beta\varepsilon e^{-\beta\varepsilon}}{1+2e^{-\beta\varepsilon}} \right) 
\end{aligned}
\end{equation}
其中$\beta = \frac{1}{kT}$,则熵与温度的关系为:
\begin{equation}\nonumber
	S = Nk\left( \ln\left( 1+2e^{-\frac{\varepsilon}{kT}} \right) + \frac{2\varepsilon e^{-\frac{\varepsilon}{kT}}}{kT(1+2e^{-\frac{\varepsilon}{kT}})} \right) 
\end{equation}
\subsection{}
系统的内能为:
\begin{equation}\nonumber
	U = -N\frac{\partial }{\partial \beta}\ln Z = \frac{2N\varepsilon e^{-\beta\varepsilon}}{1+2e^{-\beta\varepsilon}}.
\end{equation}
热熔为:
\begin{equation}\nonumber
	C = \frac{dU}{dT} = \frac{dU}{d\beta}\left(-\frac{1}{kT^2}\right) = \frac{2N\varepsilon^2}{kT^2}\frac{e^{-\frac{\varepsilon}{kT}}}{\left(1+2e^{-\frac{\varepsilon}{kT}}\right)^2}.
\end{equation}
在高温极限$\varepsilon/kT << 1$时,热熔为:
\begin{equation*}
	C = \frac{2kN}{9}\left( \frac{\varepsilon}{kT} \right)^2
\end{equation*}
则比热熔(每个粒子热熔)为:
\begin{equation}\nonumber
	c =\frac{C}{N} = \frac{2k}{9}\left( \frac{\varepsilon}{kT} \right)^2
\end{equation}
趋向于0。




\end{document}