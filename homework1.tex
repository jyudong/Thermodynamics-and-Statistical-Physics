\documentclass[a4paper,12pt]{article}

\usepackage{titlesec}
%\usepackage{geometry}
\usepackage{amsmath}   %\nonumber方程后不编号
\usepackage{xeCJK}
\setCJKmainfont{Kai}

\title{热力学统计物理第一次作业}
\date{2021.09.14}
\author{董建宇}

\begin{document}

\maketitle
%\textwidth{2mm}
\titleformat{\section}[hang]{\small}{\thesection}{0.8em}{}{}
\titleformat{\subsection}[hang]{\small}{\thesubsection}{0.8em}{}{}
\begin{section}{}
    证明:定压膨胀系数$\alpha$,定容压力系数$\beta$,等温压缩系数$\kappa_{T}$满足以下关系。p为压强
    \begin{equation}\nonumber
        \alpha=\kappa_T\beta p
    \end{equation}\\
    证明:对于一个简单系统,有:$f(p,V,T)=0$\\
    对上式做全微分得:
    \begin{equation}\nonumber
        df(p,V,T)=\frac{\partial f}{\partial p}dp+\frac{\partial f}{\partial V}dV+\frac{\partial f}{\partial T}dT=0
    \end{equation}
    则有:
    \begin{equation}\nonumber
        \left(\frac{\partial V}{\partial T}\right)_{p}=-\cfrac{\frac{\partial f}{\partial T}}{\frac{\partial f}{\partial V}},~~  
        \left(\frac{\partial p}{\partial T}\right)_{V}=-\frac{\frac{\partial f}{\partial T}}{\frac{\partial f}{\partial p}},~~ 
        \left(\frac{\partial V}{\partial p}\right)_T=-\frac{\frac{\partial f}{\partial p}}{\frac{\partial f}{\partial V}}
    \end{equation}
    由定义可知:
    \begin{equation}\nonumber
        \alpha=\frac{1}{V}\left(\frac{\partial V}{\partial T}\right)_p,~~ \beta=\frac{1}{p}\left(\frac{\partial p}{\partial T}\right)_V,~~ \kappa_T=-\frac{1}{V}\left(\frac{\partial V}{\partial p}\right)_T
    \end{equation}
    则有:
    \begin{equation}\nonumber
        \kappa_T\beta p=-\frac{1}{V}\left(\frac{\partial p}{\partial T}\right)_{V}\left(\frac{\partial V}{\partial p}\right)_T=\frac{1}{V}\left(\frac{\partial V}{\partial T}\right)_p=\alpha
    \end{equation}
    即有:
    \begin{equation}\nonumber
        \alpha = \kappa_T\beta p
    \end{equation}
\end{section}

\begin{section}{}
    实际气体的物态方程可以用以下两个公式表达:
    \begin{equation}\nonumber
        \begin{aligned}
            \left(p+\frac{N^2a}{TV^2}\right)\left(V-Nb\right)&=NRT\\
            pe^{Na/VRT}\left(V-Nb\right)&=NRT
        \end{aligned}
    \end{equation}
    其中a,b均为常数。\\
    分别针对两个公式,将压强p表达成密度N/V的函数。
    
    \begin{subsection}{}
        对于一式:
        \begin{equation}\nonumber
            \left(p+\frac{N^2a}{TV^2}\right)\left(V-Nb\right)=NRT
        \end{equation}
        有:
        \begin{equation}\nonumber
            p=\frac{N}{V}\frac{RT}{1-b\frac{N}{V}}-\frac{a}{T}\left(\frac{N}{V}\right)^2
        \end{equation}
        当$\frac{N}{V}$趋向于0时,由泰勒展开得
        \begin{equation}\nonumber
            \frac{RT}{1-b\frac{N}{V}}=RT\sum_{n=0}^{+\infty}\left(\frac{bN}{V}\right)^n
        \end{equation}
        则有:
        \begin{equation}\nonumber
            p=-\frac{a}{T}\left(\frac{N}{V}\right)^2+RT\sum_{n=0}^{+\infty}b^n\left(\frac{N}{V}\right)^{n+1}
        \end{equation}
    \end{subsection}

    \begin{subsection}{}
        对于二式:
        \begin{equation}\nonumber
            pe^{Na/VRT}\left(V-Nb\right)=NRT
        \end{equation}
        有:
        \begin{equation}\nonumber
            p=\frac{N}{V}\frac{RT}{1-b\frac{N}{V}}exp\left(-\frac{Na}{VRT}\right)
        \end{equation}
        当$\frac{N}{V}$趋向于0时,由泰勒展开得:
        \begin{equation}\nonumber
            \frac{RT}{1-b\frac{N}{V}}=RT\sum_{i=0}^{+\infty}\left(\frac{bN}{V}\right)^i
        \end{equation}
        \begin{equation}\nonumber
            exp\left(-\frac{a}{RT}\frac{N}{V}\right)=\sum_{j=0}^{\infty}\frac{1}{j!}\left(-\frac{a}{RT}\frac{N}{V}\right)^j
        \end{equation}
        则有:
        \begin{equation}\nonumber
            p=RT\sum_{i=0}^{\infty}b^i\left(\frac{N}{V}\right)^{i+1}\sum_{j=0}^{\infty}\frac{1}{j!}\left(-\frac{a}{RT}\frac{N}{V}\right)^j
        \end{equation}
    \end{subsection}
\end{section}

\begin{section}{}
    已知某系统$\alpha=\frac{1}{T}\left(1+\frac{3a}{VT^2}\right)$,$\kappa_T=\frac{1}{p}\left(1+\frac{a}{VT^2}\right)$,证明该系统物态方程为:
    \begin{equation}\nonumber
        pV=bT-\frac{ap}{T^2}
    \end{equation}
    其中b为常数。\\
    \\
    证明:假设$V=V(p,T)$,有定义可知:
    \begin{equation}\nonumber
        \frac{\partial V}{\partial p}=-V\kappa_T=-\frac{V}{p}-\frac{a}{pT^2}
    \end{equation}
    则有:
    \begin{equation}\nonumber
        \frac{dV}{V+\frac{a}{T^2}}=-\frac{1}{p}\,dp
    \end{equation}
    两侧对$p$积分得:
    \begin{equation}\nonumber
        \ln\left(V+\frac{a}{T^2}\right)=-\ln p+\ln C(T)
    \end{equation}
    即:
    \begin{equation}\nonumber
        V=\frac{C(T)}{p}-\frac{a}{T^2}
    \end{equation}
    其中,$C(T)$为只关于$T$的函数。\\
    有定义可知:
    \begin{equation}\nonumber
        \left(\frac{\partial V}{\partial T}\right)_p=\frac{C'(T)}{p}+\frac{2a}{T^3}=V\alpha=\frac{V}{T}+\frac{3a}{T^3}
    \end{equation}
    整理可得:
    \begin{equation}\nonumber
        TC'(T)=C(T)
    \end{equation}
    即$C(T)=bT$,其中,b为常数。\\
    则有:
    \begin{equation}\nonumber
        pV=bT-\frac{ap}{T^2}
    \end{equation}
\end{section}

\end{document}