\documentclass[a4paper,12pt]{article}

\usepackage{titlesec}
%\usepackage{geometry}
\usepackage{amsmath}   %\nonumber方程后不编号
\usepackage{xeCJK}
\setCJKmainfont{Kai}

\title{热力学统计物理第5次作业}
\date{2021.10}
\author{董建宇 ~ 2019511017}

\begin{document}

\maketitle

\titleformat{\section}[hang]{\small}{\thesection}{0.8em}{}{}
\titleformat{\subsection}[hang]{\small}{\thesubsection}{0.8em}{}{}

\section{}
由克拉伯龙方程可知:
\begin{equation}\nonumber
	\frac{\,dp}{\,dT} = \frac{L}{T\left(V_m^{\beta} - V_m^{\alpha}\right)}
\end{equation}
其中$\alpha$表示液相,$\beta$表示气相,将水蒸气看作理想气体,则有$V_m^{\beta} \gg V_m^{\alpha}$,即可在近似条件下忽略$V_m^{\alpha}$。同时有$pV_m^{\beta} = RT$,则克拉伯龙方程可化为:
\begin{equation}\nonumber
	\frac{1}{p}\frac{\,dp}{\,dT} = \frac{L}{RT^2}
\end{equation}
可以解得:
\begin{equation}\nonumber
	\ln p = -\frac{L}{RT} + A
\end{equation}
其中A为积分常数。则有:
\begin{equation}\nonumber
	L = \frac{R\left(\ln p_1 - \ln p_2\right)}{\frac{1}{T_2} - \frac{1}{T_1}} = 4.13\times 10^4 J/mol
\end{equation}


\section{}
由题目1可知,近似有:
\begin{equation}\nonumber
	\ln (p/mmHg) = -\frac{L}{RT} + A
\end{equation}
近似条件为假设相变浅热不随温度变化,则在计算某温度的相变浅热时,尽可能选择在该温度附近的条件进行计算,则278K时的汽化热为:
\begin{equation}\nonumber
	L = \frac{R \ln \left(\frac{p_1}{p_3}\right)}{\frac{1}{T_3} - \frac{1}{T_1}}
\end{equation}
其中$p_1 = 5.69mmHg, p_3 = 7.71mmHg, T_1 = 276K, T_3 = 280K$ \\
计算可得:$L = 4.88\times 10^4 J/mol$


\section{}
由题目可知,在700K到739K范围内,镁的饱和蒸气压p与T的关系为:
\begin{equation}\nonumber
	\lg (p/mmHg) = -\frac{7527}{T} + 8.589
\end{equation}
注意到:
\begin{equation}\nonumber
	\ln (p/mmHg) = \frac{\lg (p/mmHg)}{\lg e} = -\frac{7527}{T \lg e} + \frac{8.589}{\lg e}
\end{equation}
则有
\begin{equation}\nonumber
	L = \frac{R \times 7527(K)}{\lg e} = 1.441\times 10^5 J/mol
\end{equation}


\section{}
设水的相变浅热不随温度变化,则液态水的蒸汽压方程为:
\begin{equation}\nonumber
	\ln p = -\frac{L}{RT} + A
\end{equation}
已知100$^{\circ}C$时水的汽化热为$L_1 = 4.13\times 10^4 J/mol$,可以计算得$A = 24.81$ \\
则当温度为$15^{\circ}C$时水的饱和蒸气压为:
\begin{equation}\nonumber
	p' = 1942.2 N/m^2
\end{equation}



\section{}
\subsection{}
联立固态氨蒸气压方程与液态氨蒸汽压方程得:
\begin{equation}\nonumber
\left\{\begin{aligned}
	\ln (p/mmHg) &= 23.3 - \frac{3754}{T} \\
	\ln (p/mmHg) &= 19.49 - \frac{3063}{T}
\end{aligned}\right.
\end{equation}
可以解得:
\begin{equation}\nonumber
\left\{\begin{aligned}
	p &=  13.48mmHg\\
	T &= 181.4K
\end{aligned}\right.
\end{equation}
即三相点的压强为$13.48mmHg$,温度为$181.4K$
\subsection{}
汽化热为:
\begin{equation}\nonumber
	L_1 = R \times 3063(K) = 2.547\times 10^4J/mol
\end{equation}
升华热为:
\begin{equation}\nonumber
	L_2 = R \times 3754(K) = 3.121\times 10^4 J/mol
\end{equation}
熔化热为:
\begin{equation}\nonumber
	L_3 = L_2 - L_1 = 5.74\times 10^3 J/mol
\end{equation}


\end{document}