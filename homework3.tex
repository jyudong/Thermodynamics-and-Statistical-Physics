\documentclass[a4paper,12pt]{article}

\usepackage{titlesec}
%\usepackage{geometry}
\usepackage{amsmath}   %\nonumber方程后不编号
\usepackage{xeCJK}
\setCJKmainfont{Kai}

\title{热力学统计物理第三次作业}
\date{2021.09.28}
\author{董建宇}

\begin{document}

\maketitle

\titleformat{\section}[hang]{\small}{\thesection}{0.8em}{}{}
\titleformat{\subsection}[hang]{\small}{\thesubsection}{0.8em}{}{}

\section{}
\subsection{}
设水的比热容为常数$C=4.18\times 10^{3}J/(kg~^{\circ}C)$,则水温度变化dT时吸热为:
\begin{equation}\nonumber
	dQ=mC\,dT
\end{equation}
则水的熵变为:
\begin{equation}\nonumber
	\Delta S_1=\int_{T_1}^{T_2}\frac{dQ}{T}=\int_{T_1}^{T_2}\frac{mC\,dT}{T}=1303.99 J/^{\circ}C
\end{equation}
\subsection{}
将水与热源视作一个整体,由能量守恒得热源放热为:
\begin{equation}\nonumber
	\Delta Q=-mC(T_2-T_1)=-4.18\times 10^{5}J
\end{equation}
则热源熵变为:
\begin{equation}\nonumber
	\Delta S_2=\frac{\Delta Q}{T_2}=-1120.19 J/K
\end{equation}
则整个系统熵变为:
\begin{equation}\nonumber
	\Delta S=\Delta S_1+\Delta S_2=183.80 J/K
\end{equation}
\subsection{}
欲使整个系统熵变为0,可以设计无穷个热源温度连续分布在$0^{\circ}C$到$100^{\circ}C$,使$0^{\circ}C$的水与温度连续分布的热源接触,则热源此时的熵变为:
\begin{equation}\nonumber
	\Delta S_3=\int_{T_1}^{T_2}\frac{dQ'}{T}=-\int_{T_1}^{T_2}\frac{dQ}{T}=-\Delta S_1
\end{equation}
系统熵变为:
\begin{equation}\nonumber
	\Delta S'=\Delta S_1+\Delta S_3=0
\end{equation}

\section{}
引入状态三:$-196^{\circ}C$,压力为1atm的1g氮气,则有从状态二($-196^{\circ}C,压力为1atm的1g液氮$)到状态三的熵变为:
\begin{equation}\nonumber
	\Delta S_1=\frac{\Delta Q_1}{T_1}=\frac{\mu m}{T_1}=0.617 cal/K
\end{equation}
从状态三到状态一过程,熵变为:
\begin{equation}\nonumber
	\Delta S_2=\int_{T_1}^{T_2}\frac{dQ}{T}=\int_{T_1}^{T_2}\frac{mc_p\,dT}{MT}=0.334 cal/K
\end{equation}
则两种状态的熵差为:
\begin{equation}\nonumber
	S_1-S_2=\Delta S_1+\Delta S_2=0.951 cal/K
\end{equation}

\section{}
\subsection{}
单原子理想气体内能为:
\begin{equation}\nonumber
	U=c_VNT
\end{equation}
由热力学第一定律可知:
\begin{equation}\nonumber
	dQ=dU+pdV=Nc_V\,dT+p\,dV
\end{equation}
熵变为:
\begin{equation}\nonumber
	\Delta S=\int\frac{dQ}{T}=Nc_V\int\frac{\,dT}{T}+nR\int\frac{\,dV}{V}=Nc_V\ln T+nR\ln V +C
\end{equation}
其中,C为常数。\\
则单原子气体的自由能为:
\begin{equation}\nonumber
	F=U-TS=\frac{3}{2}NRT-T\left(S_0+\frac{3}{2}NR\ln T+nR\ln V +C\right)
\end{equation}
其中$S_0$为系统初始的熵。
\subsection{}
由理想气体状态方程可知,初始状态满足:
\begin{equation}\nonumber
	p_i^{I}V_i^{I}=nRT_i^{I},~p_i^{II}V_i^{II}=nRT_i^{II}
\end{equation}
活塞经过可逆移动后有:
\begin{equation}\nonumber
	p_f^{I}V_f^{I}=nRT_f^{I},~p_f^{II}V_f^{II}=nRT_f^{II}
\end{equation}
由于容器器壁是导热的,且容器放在了温度为$0^{\circ}C$的“热池”中,则有:
\begin{equation}\nonumber
	T_f^{I}=T_f^{II}=T_0=0^{\circ}C
\end{equation}
由于活塞的移动是可逆的,则这是一个准静态过程,即任意时刻两部分气体压强差都相等,且都等于$p_f^{II}-p_f^{I}$。
则系统对外做功为:
\begin{equation}\nonumber
	W=\left(p_f^{II}-p_f^{I}\right)\left(V_i^I-V_f^I\right)=nRT_0\left(\frac{1}{V_f^{II}}-\frac{1}{V_f^{I}}\right)\left(V_i^I-V_f^I\right)=302.81J
\end{equation}





\end{document}
